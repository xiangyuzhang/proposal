\documentclass{article}
\usepackage[a4paper, total={6in, 8in}]{geometry}
\usepackage{setspace}
\doublespacing
\begin{document}

\onecolumn

{ Dear Editor and Reviewers,}
\vspace{5mm}
%{ Cunxi Yu, Walter Brown, Duo Liu, Andre Rossi, Maciej Ciesielski}

{ This papers is an extended version of our conference paper ["\textbf{Oracle-Guided Incremental SAT Solving to Reverse Engineer Camouflaged Logic Circuits}]". D. Liu, C. Yu, X. Zhang, D. Holcomb] published in 2016 Design, Automation and Test in Europe conference (DATE'16), attached with this submission. The following is a list of new contributions comparing to the DATE-16 paper.}
\vspace{5mm}


{1. We present an incremental-SAT-based algorithm for reverse engineering camouflaged integrated circuits that outperforms the best existing reverse engineering algorithms by 10.5x on ISCAS-85 benchmarks. We provide the extensive comparison between baseline algorithm and our incremental SAT-based algorithm (Figure 11, 13).}


{2. In this paper, we provide a new standard and widely-applicable tool for logic deobfuscation that can be used to evaluate current and future approaches for selective camouflaging. The source code for our tool is included in the paper for review (link included in Section 1).}


{3. We demonstrate that our technique is general and can efficiently resolve the obfuscated function of three proposed camouflaging techniques [2], [3], and [4].}


{4. We show that selective gate camouflaging based on the objective of maximizing output corruption [2] offers no resistance to reverse engineering and can reduce the number of vectors required to deobfuscate a circuit (Figure 10).}


{5. The paper provides a substantially more extensive description and analysis of the incremental SAT-based algorithm, including two step-by-step solving examples (Figures 6, 7, 8).}


\vspace{5mm}
Best regards,

Cunxi Yu
\end{document}
